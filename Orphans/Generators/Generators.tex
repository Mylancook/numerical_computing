\lab{Generators}{Generators}
\label{lab:Python_Generators}

Lists, tuples, sets, and dictionaries are examples of sequences in Python.
Often it is useful to visit all the elements of a sequence.  This process of visiting
each element of a sequence once is called \emph{iteration}. 
Each of the Python types we have mentioned define their own iterators.
You use them every time you execute the statement \li{for <elem> in <sequence>}.

Python also has a special type of object called a \emph{generator}.
Similar to an iterator, a generator returns a sequence of values.
The important difference is that a generator computes the next value in the sequence and returns it.
It never has to store all of the values in the sequence.
A good way to think of the distinction between lists and generators is to remember: \emph{Iterators return their values while generators yield their values}.

Python uses the \li{yield} keyword to define a generator.  This is perhaps best illustrated using an example.  Consider the following, where we re-implement Python's \li{range} function as an iterator and a generator.
\begin{lstlisting}
def range_iter(start, stop, step=1):
    i = start
    r = []
    while i < stop:
       r.append(i)
       i = i + step
    return r
\end{lstlisting}
\begin{lstlisting}
def range_gen(start, stop, step=1):
    i = start
    while i < stop:
        yield i
        i = i + step
\end{lstlisting}
The two functions look very similar.  But, you will soon see that they behave very differently.
The first function, when executed will immediately build a list, one element at a time until the done
at which point it returns the entire list.  On the author's computer, this function takes about 1.43ms
to build and return a list of 10000 elements.  The second function is only marginally better at
returning the 10000 elements in approximately 1.09ms.  
This gap between the two functions grows much wider as the inputs are increased.

Where the real difference in the two functions lies is in the time the function takes to return (or yield) its first value.
The first function requires 1.43ms to return its first value, because it waits until it has computed the entire list and then returns the whole thing at once.
The second function takes a mere .00000041ms to yield its first value because it only creates a \emph{generator} object.
This object has several methods.
The most important and most useful method is the \li{next()} method.  Every time this method is called, 
the generator resumes from its previous state and computes the next value in the sequence. Each time \li{yield} statement is executed, the generator is effectively suspended until \li{next()} is called again.
When the generator has finished executing, a \li{StopIteration} exception is raised and the generator terminates.
You can send values and throw exceptions to the generators via the \li{send()} and \li{throw()} methods respectively.
For more information on these methods we refer to PEP 342 (Python Enhancement Proposal 342).

When you are writing \li{for} loops, you should use a generator whenever it is reasonable to do so.
The \li{xrange()} function that is built into Python is a generator based version of \li{range()} and is generally faster when used in \li{for} loops.

\section*{Combinations}
Combinations are subsets of a set.  The power set of a set, $S$, is the set of all combinations
of elements in $S$.  The cardinality of the the power set is $2^{\abs{S}}$.  Clearly, if our
set is of any appreciable size, the power set will be much larger.  Let's look at one method for
computing the power set of a set.

We will use a convenient correspondence between elements of the power set and binary numbers.
If $S$ has $n$ elements, then we can use an $n$-bit binary integer to represent each subset.
If the $i$th element of $S$ is in the subset, then the $i$th bit of the integer is a $1$.
Otherwise it is a $0$.
For example, if $S = \{a,b,c,d,e\}$, then the subset $\{b,d,e\}$ corresponds to the integer $01011$.

Using this representation, we can count through the power set of any finite set $S$, by counting from 0 to $2^{\abs{S}}-1$, looking at the binary representation of the number, and then including or excluding elements based on the value of each bit.

\begin{problem}
Write a function that will generate a list of the first $n$ binary numbers.  This should accept an integer $n$ and return a list of strings.
\end{problem}

\begin{problem}
Write a function that will accept a list and return a list of all the combinations
of elements of that list.
\end{problem}

This is, however, an inefficient way to compute combinations.
It is faster to make each combination by modifying the previous combination, especially if the modification can be made minimal.
However, with
our current method of counting in binary, we are rebuilding the combination from scratch
each time.  For example, if we have a combination represented by \texttt{01111}, the next
combination would be \texttt{10000}.  We added one element and removed four elements!

Fortunately, there is a really neat way of avoiding such large modifications ---  Gray codes. 
A Gray code is a sequence of binary numbers where each new number is generated by
changing the previous code by exactly one bit.  Geometrically, we can think of it as
traversing a unit $n$-cube by moving only along the edges of the cube.
Gray codes are used frequently in error correction schemes.

A Gray code is constructed as follows.\cite{Brualdi2010}
The code of order $n$ can be calculated recursively in the following way:
\begin{enumerate}
\item The Gray code of order 1 is 0, 1.
\item Compute the Gray code of order $n-1$.  Write the binary numbers in a new list and then write them again in reverse order on the end of the same list (reflect them so that the last Gray code or order $n-1$ is first and the first Gray code of order $n-1$ is last).
\item Prepend a 0 to the first half of the list and prepend a 1 to the remaining half of the list.
\end{enumerate}
While this is a simple algorithm to describe, it is not very efficient.  
To generate a Gray code of order $n$, we have to generate all previous Gray codes.
Fortunately, there exists another way to compute Gray codes of order $n$ more efficiently.
The algorithm is given in Brualdi's book.  
We first note that a Gray code of order $n$ is of length $2^n$ with each code of length $n$.
We also observe that the reflected gray codes always begin with $0\dots0$ and terminate with $10\dots0$.
\begin{enumerate}
\item Start with $0\dots0$ ($n$ zeros).  This is the current binary number.
\item Sum the digits of the current binary number.
\begin{enumerate}
\item If the sum of the digits is even, flip the last bit.
This becomes our new current number. Go to Step 2.
\item If the sum of the digits is odd, find the rightmost non-zero bit and flip the bit immediately to the left of it.
This becomes our new current number. Go to Step 2.
\item If the rightmost non-zero bit of the number is the first one you have reached the end of the Gray code.
\end{enumerate}
\end{enumerate}

\begin{problem}
\label{prob:brualdi_gray}
Implement the second algorithm above for calculating the Gray code of order $n$.
Your implementation must function as a generator.
\end{problem}

\begin{problem}
\label{prob:changed_elem_gray}
Gray codes are an ordering of the power set such that any consecutive subsets differ by exactly one element.  Write a generator that will only return the change required to arrive at the next subset in the ordering.  Each time the generator is called, 
it should return a single element and a boolean value that determines if that element should be
added or removed from the set to obtain the next Gray code.  This function should be able to accept any set with arbitrary elements.
\end{problem}

\printbibliography