\lab{Linear Transformations}{Linear Transformations}
% \objective{Apply affine transformations to a set of vectors in $\mathbb{R}^2$ and solve linear systems.}
% \objective{Introduce the temporal and spatial complexity and explore SciPy's methods for working with sparse matrices.}

\section*{SciPy} % ============================================================

SciPy is . . .

\subsection*{Linear Algebra} % ------------------------------------------------
Both NumPy and SciPy have a linear algebra library, but the SciPy library is larger.
The SciPy linear algebra library is typically imported as follows:

\begin{lstlisting}
from scipy import linalg as la
\end{lstlisting}

The linear algebra library contains several functions to construct special
matrices, located in \li{linalg.special_matrices}.
There are also functions that will invert matrices, find determinants and norms, solve linear systems and least squares problems, and find special matrix decompositions.
You can read more about the linear algebra capabilities of SciPy in the documentation for the \li{linalg} module found at \url{http://docs.scipy.org/doc/scipy/reference/linalg.html}.

Finally, the \li{scipy.linalg} library has a \li{matrix} class that is very
similar to a 2-D NumPy array.
The matrix class can be convenient when doing matrix operations.
However, in such situations we still recommend using NumPy arrrays, which have many of the same features and are also compatible with all other SciPy operations.

The \li{scipy.linalg} library will be essential for the remainder of the labs in this manual.
We will address the details of this package at length in future labs.
