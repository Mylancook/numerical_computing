%\newcommand{\of}{\texttt{.o}~}

\lab{Interfacing With C++ Using Cython}{Interfacing With C++ Using Cython}
\label{lab:cythonwrap_cpp}

\objective{Learn to interface with object files using Cython. This lab should be worked through on a machine that has already been configured to build Cython extensions using gcc or MinGW.}

\section*{Interfacing with C++}
Cython can also be used to wrap C++ functions and classes.
It can compile entire modules to C++ instead of C if we specify C++ as the language in the setup file.
Here we will focus primarily on wrapping C++ classes since wrapping a C++ function is roughly the same as wrapping a function from C or Fortran.
Cython also supports wrapping templated classes and functions.
Many other C++ features like operator overloading and the \li{new} and \li{del} operators are also supported when using C++.

The process for exposing a C++ class to Cython is essentially the same.
The C++ code must first be compiled to an object file.
In our Cython file we then declare that we will be importing functions and classes from a C++ file so that the Cython compiler knows how they will behave.
We then compile our Cython file by first compiling to C++

Included with this lab are three C++ classes.
The \li{Permutation} class is the primary class we will be wrapping.
It implements arithmetic for the Cauchy cycle representation of permutations.
For those not familiar with permutation arithmetic, there are some examples in the file \li{unittest.py}.
The other two classes are used in the construction of the \li{Permutation} class.
Because the Permutation class depends on the \li{Node} and \li{Cycle} classes, they must all be compiled together.
Here is the \li{setup.py} to set up the wrapper for the \li{Permutation} class.

\lstinputlisting[style=fromfile, language=Python, name=pypermutation_setup.py]{./permutations/pypermutation_setup.py}

\begin{warn}
When wrapping C++ classes, templates, etc. you should always make sure that you specify the language as C++ as we have shown in this setup file.
\end{warn}

We need to tell the Cython compiler how to interface with the C++ class.
This can be done in much the same way that we defined the interfaces for C and Fortran functions, but we now need to declare that we are importing a class with several functions that act on it.
We will include the declarations corresponding to this class in a \li{.pxd} file.
\li{.pxd} are called \emph{declaration} files.
Though declarations are, strictly speaking, not ``Cython headers'', they can be used in similar ways.
Here we will declare the Permutation class and some of its public methods so that we can use them in Cython.
\lstinputlisting[style=fromfile, language=Python, name=Permutation.pxd]{./permutations/Permutation_incomplete.pxd}
This \li{.pxd} file can be included in our Cython file with the line \li{from Permutation cimport Permutation}.
Cython comes with several pre-built declaration files.
These declaration files are good examples of the proper syntax for wrapping C++ class templates.
Some of these declaration files allow Cython code to interface with containers and other classes from the C++ standard library like \li{deque}, \li{list}, \li{map}, \li{pair}, \li{queue}, \li{set}, \li{stack}, \li{string}, \li{utility}, and \li{vector}.
Each of these declaration files can be loaded by using something like
\begin{lstlisting}
from libcpp.vector cimport vector
\end{lstlisting}
Cython also supports some kinds of automatic conversion from Python data types to standard library containers.
See the Cython documentation for more information on interfacing with C++ containers.

As we mentioned before, the general approach for wrapping a C++ class is to make a Cython class that wraps a pointer to the C++ class.
Here is an example of how this is done.
\lstinputlisting[style=fromfile, language=Python, name=pypermutation.pyx]{./permutations/pypermutation_incomplete.pyx}
Notice that we included the declaration file for the class.

\begin{problem}
The Permutation class includes methods to trace an index through the inverse permutation, get the minimum index that is changed by the permutation, and raise the permutation to a power.
The declarations for these functions are commented out in the \li{pypermutation.pyx} file.
Uncomment the declarations and add in the code necessary so that these methods are also available in Python.
These functions are declared in the header \li{Permutation.hpp}, but they are not declared in the declaration file \li{Permutation.pxd}.
Add the proper declarations to the declaration file.

Note: the argument $z$ that is included in the power function for the Permutation class is a mandatory argument for the exponentiation operator in Cython.
It tells the computer to compute the power of a number mod another number.
In this case, raise an error if $z$ is not \li{None}.

Look at the \li{__mul__} method for an example of how to construct and return a new PyPermutation object.
Keep in mind that the \li{power} method of the \li{Permutation} class in C++ returns a pointer to a new \li{Permutation} object.
\end{problem}

%\let\of\undefined
