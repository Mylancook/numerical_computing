\lab{Introduction to Python}{Introduction to Python}
\label{lab:Python_Intro}
\objective{Python is a powerful general-purpose programming language.
It can also be used interactively, allowing for very rapid development.
Python has many powerful scientific computing tools, making it an ideal language for applied and computational mathematics.
\\ \indent
In this introductory lab we introduce Python syntax, data types, functions, and control flow tools.
These Python basics are an essential part of almost every problem you will solve and almost every program you will write.
}

% FUTURE EDITS
%  - transition from Python 2.7 to Python 3.5
%     - Use parenthesis with all print statements, as print is now a function.

\section*{Getting Started} % ==================================================

% TODO: Update to Python 3
Python is quickly gaining momentum as a fundamental tool in scientific computing.
It is simple, readable, and easy to learn.
\emph{Anaconda} is a free distribution service by Continuum Analytics, Inc., that includes the cross-platform Python \emph{interpreter} (the software that actually executes Python code) and many Python libraries that are commonly used for applied and computational mathematics.
To install Python via Anaconda, go to \url{http://continuum.io/downloads}, download the installer corresponding to your operating system, and follow the on-screen instructions.

% TODO: Change this for Python 3
Use the installer corresponding to \textbf{Python 2.7}.
Although later versions of Python are available, they do not yet have many of the libraries and features that we require in this curriculum.
For more information on installing Python and various libraries, see Appendices \ref{pythoninstall} and \ref{updateinstall}.

\subsection*{Running Python} % ------------------------------------------------

Python files are saved with a \texttt{.py} extension.
For beginners, we strongly recommend using a simple text editor\footnote{Some of the more popular text editors are described in Appendix \ref{pythoninstall}.} for writing Python files, though many free IDEs (Integrated Development Environments---large applications that facilitate code development with some sophisticated tools) are also compatible with Python.
For now, the simpler the coding environment, the better.

A plain Python file looks similar to the code on the following page:

\begin{lstlisting}
# filename.py
"""This is the file header.
The header contains basic information about the file.
"""

if __name__ == "__main__":
    pass                            # 'pass' is a temporary placeholder.
\end{lstlisting}

The \li{#} character creates a single-line \emph{comment}.
Comments are ignored by the interpreter and serve as annotations for the accompanying source code.
A pair of three quotes, \li{""" """} or \li{''' '''}, creates a multi-line string literal, which may be used as a multi-line comment.
% Comments on the very first line can be used to give the interpreter special instructions, which we will discuss briefly in Volume III.
% For now, we put the name of the file on the first line.

A triple-quoted string literal at the top of the file serves as the \emph{header} for the file.
The header typically includes the author, instructions on how to use the file, and so on.
Executable Python code comes after the header.

Finally, each statement under the \li{if __name__ == "__main__"} clause is executed when the file is run.
Any ``orphan'' statements that are not placed under this clause are also executed when the file is run, but it is generally considered bad practice to leave any statements outside of function definitions or under the clause.

\begin{problem} % Problem: Run a file that prints "Hello, world!"
Create a new Python file and save it to your file system.
Write the file name and a brief header at the top, then add the following code:

\begin{lstlisting}
if __name__ == "__main__":
    print("Hello, world!")
\end{lstlisting}

Open a Unix shell (often called a \emph{terminal} or a \emph{command line}) and navigate to the directory where the new file is saved.
Use the command \li{ls} to list the files in the current directory, \li{pwd} to print the working directory, and \li{cd} to change directories.
Below, the \li{\$} character indicates that the command is being run from the shell.

\begin{lstlisting}[language=bash]
$ pwd
/Users/Guest
$ ls
Desktop     Documents       Downloads       Pictures        Music
$ cd Documents
$ pwd
/Users/Guest/Documents
$ ls
python_intro.py
\end{lstlisting}

Now that we are in the same directory as the Python file, we can execute it using the following command:

\begin{lstlisting}
$ python python_intro.py
\end{lstlisting}

If ``Hello, world!'' is displayed on the screen, you have just successfully executed your first Python program!
\label{prob:helloWorld}
\end{problem}

\subsection*{IPython} % -------------------------------------------------------

Python can be run interactively using several interfaces.
The most basic of these is the Python interpreter.
In this and subsequent labs, the triple brackets \li{>>>} indicate that the given code is being executed one line at a time via the Python interpreter.

\begin{lstlisting}
$ python                            # Start the Python interpreter.
>>> print("This is plain Python.")  # Execute some code.
<<This is plain Python.>>
\end{lstlisting}

There are, however, more useful interfaces.
Chief among these is \emph{IPython}.
To execute a script in IPython, use the \li{run} command.

\begin{lstlisting}
>>> quit()                          # Exit the Python interpreter.
$ ipython                           # Start IPython.

In [1]: print("This is IPython!")   # Execute some code.
<<This is IPython!>>

In [2]: run python_intro.py         # Run a particular Python script.
Hello, world!
\end{lstlisting}

One of the biggest advantages of IPython is that it supports \emph{object introspection}, whereas the regular Python interpreter does not.
Object introspection allows you to quickly see all methods and attributes associated with an object.
IPython also has a built-in \li{help()} function that provides interactive help.

\begin{lstlisting}
# A list is a basic Python data structure. To see the methods associated with
# a list, type the object name (list), followed by a period, and press tab.
In [1]: list.   # Press 'tab'.
<<list.append   list.extend   list.insert   list.remove      list.sort
list.count    list.index    list.pop      list.reverse>>

# To learn more about a specific method, use "?".
In [1]: list.append?
<<Docstring:    L.append(object) -- append object to end
Type:         method_descriptor>>

In [2]: help()                      # Start IPython's interactive help utility.

<<help>>> list                          # Get documentation on the list class.
<<Help on class list in module __builtin__:

class list(object)
 |  list() -> new empty list
 |  list(iterable) -> new list initialized from iterable's items
 |  ...>>                             # Press 'q' to exit the info screen.

<<help>>> quit                          # End the interactive help session.
\end{lstlisting}

Use IPython side-by-side with a text editor to test syntax and small code snippets.
Testing code in IPython \emph{before} putting it into a program helps you catch errors beforehand and greatly speeds up the coding process.

If you cannot answer a coding question using these strategies, search the Internet.
In particular, \href{http://stackoverflow.com/}{stackoverflow.com} is often a valuable resource for answering common programming questions.

\section*{Python Basics} % ====================================================

\subsection*{Arithmetic} % ----------------------------------------------------

We can now dive into the particulars of Python syntax.
To begin, Python can be used as a calculator.
Use \li{**} for exponentiation and \li{\%} for modular division.

\begin{lstlisting}
>>> 3**2 + 2*5                      # Python obeys the order of operations.
19

>>> 13 % 3                          # The modulo operator % calculates the
1                                   # remainder: 13 = (3*4) + 1.
\end{lstlisting}

In most Python interpreters, the underscore character \li{\_} is a variable with the value of the previous command's output, like the ANS button on many calculators.

\begin{lstlisting}
>>> 12 * 3
36
>>> _ / 4
9
\end{lstlisting}

Data comparisons like \li{<} and \li{>} act as expected.
The \li{==} operator checks for numerical equality and the \li{<=} and \li{>=} operators correspond to $\le$ and $\ge$, respectively.

To connect multiple boolean expressions, use the operators \li{and}, \li{or}, and \li{not}.\footnote{In many other programming languages, the \li{and}, \li{or}, and \li{not} operators are written as \li{&&}, \li{||}, and \li{!}, respectively. Python's convention is much more readable and does not require parentheses.}

\begin{lstlisting}
>>> 3 > 2
True
>>> 4 < 4
False
>>> 1 <= 1 or 2 > 3
True
>>> 7 == 7 and not 4 < 4
True
\end{lstlisting}

\subsection*{Variables} % -----------------------------------------------------

Variables are used to temporarily store data.
A \emph{single} equals sign assigns one or more values (on the right) to one or more variable names (on the left).
A \emph{double} equals sign is a comparison operator, as in the previous code block.

Unlike many programming languages, Python does not require a variable's data type to be specified upon initialization.

\begin{lstlisting}
>>> x = 12                          # Initialize x with the integer 12.
>>> y = 2 * 6                       # Initialize y with the integer 2*6 = 12.
>>> x == y                          # Compare the two variable values.
True

>>> x, y = 2, 4                     # Give both x and y new values in one line.
>>> x == y
False
\end{lstlisting}

\subsection*{Functions} % -----------------------------------------------------

To define a function, use the \li{def} keyword followed by the function name, a parenthesized list of formal parameters, and a colon.
Then indent the function body.
Many other languages use the curly braces \{ and \} to delimit blocks, but Python uses whitespace indentation.
This should be done using exactly \textbf{four} spaces.
Most text editors will allow you to set the indentation type to spaces so you can use the tab key on your keyboard to insert four spaces.

\begin{lstlisting}
# No indentation leads to an error.
>>> def add(x, y):
...result = x + y
<<IndentationError: expected an indented block>>

# Instead, indent all lines that should be included in the function.
>>> def add(x, y):
...    result = x + y
...    return result
\end{lstlisting}

Functions are defined with \emph{parameters} and called with \emph{arguments}, though the terms are often used interchangeably.
Below, \li{width} and \li{height} are parameters for the function \li{area()}.
The values \li{2} and \li{5} are the arguments that are passed when calling the function.

\begin{lstlisting}
>>> def area(width, height):
...     return width * height
...
>>> area(2, 5)
10
\end{lstlisting}

It is also possible to specify default values for a function's formal parameters.
In the following example, the function \li{printer()} has three formal parameters, and the value of \li{c} defaults to 0.
That is, if it is not specified in the function call, the variable \li{c} will contain the value 0 when the function is executed.

\begin{lstlisting}
>>> def printer(a, b, c=0):
...    print a, b, c
...
>>> printer(1, 2, 3)                # Specify each parameter.
1 2 3
>>> printer(1, 2)                   # Specify only non-default parameters.
1 2 0
\end{lstlisting} % TODO Python 3: give the print statement parentheses

Arguments are passed to functions based on position or name, and positional arguments must be defined before named arguments.
For example, \li{a} and \li{b} must come before \li{c} in the function definition of \li{printer()}.
Examine the following code blocks demonstrating how positional and named arguments are used to call a function.

\begin{lstlisting}
# Try defining printer with a named argument before a positional argument.
>>> def printer(c=0, a, b):
...    print a, b, c
...
SyntaxError: non-default argument follows default argument
\end{lstlisting}

\begin{lstlisting}
# Correctly define printer() with the named argument after positional arguments.
>>> def printer(a, b, c=0):
...    print a, b, c
...

# Call printer() with 3 positional arguments.
>>> printer(2, 4, 6)
2 4 6

# Call printer() with 3 named arguments. Note the change in order.
>>> printer(b=3, c=5, a=7)
7 3 5

# Call printer() with 2 named arguments, excluding c.
>>> printer(b=1, a=2)
2 1 0

# Call printer() with 1 positional argument and 2 named arguments.
>>> printer(1, c=2, b=3)
1 3 2
\end{lstlisting}

Python functions can return more than one value.
Simply separate the values by commas after the \li{return} statement.

\begin{lstlisting}
>>> def arithmetic(a, b):
...     difference = a - b
...     product = a * b
...     return difference, product
...
>>> x, y = arithmetic(5, 2)         # x is the difference, y is the product.
>>> print x, y
3 10
\end{lstlisting}

\begin{info} % Note on the difference between print and return.
The functions \li{area()} and \li{printer()} defined above have an important difference: \li{area()} ends with a \li{return} statement, but \li{printer()} lacks a \li{return} statement.

The \li{return} statement instantly ends the function call and passes the return value to the function caller.
The \li{print} statement does nothing more than display the value of a given object (or objects) in the terminal.
% TODO: print is a function in Python 3.
A function without a return statement implicitly returns the Python constant \li{None}, which is similar to the special value \li{null} of many other languages.

\begin{lstlisting}
>>> x = area(2, 4)              # Calling area() doesn't print anything,
>>> y = printer(1, 2, 3)        # but calling printer() does.
1 2 3

>>> print(x)                    # However, x now contains a value,
8
>>> print(y)                    # whereas y only contains 'None'.
None
\end{lstlisting}
%
% If you have any intention of using the results of a function, always use a \li{return} statement.
% The returned value or values can then be stored in variables for later use.
\end{info}

\begin{problem} % Problem: write a very basic function.
In your Python file from Problem \ref{prob:helloWorld}, define a function called \li{sphere_volume()}.
This function should accept a single parameter \li{r} and return (not print) the volume of the sphere of radius \li{r}.
For now, use $3.14159$ as an approximation for $\pi$.

To test your function, call it under the \li{if __name__ == "__main__"} clause and print the returned value.
Run your file to see if your answer is what you expect it to be.
\end{problem}

\section*{Data Types} % =======================================================

\subsection*{Numerical Types} % -----------------------------------------------

Python has four numerical data types: \li{int}, \li{long},
\li{float}, and \li{complex}.
Each stores a different kind of number.
The built-in function \li{type()} identifies an object's data type.

\begin{lstlisting}
>>> type(3)                         # Numbers without periods are integers.
<<int>>

>>> type(3.0)                       # Floats have periods (3. is also a float).
<<float>>
\end{lstlisting}

% TODO: Update this section for Python 3.
Python has two types of division: integer and float.
% The \li{/} operator performs float division (fractional division), and the \li{//} operator performs integer division.
% Integer division rounds the result down to the next integer.
Integer division rounds the the result down to the next integer, and float division performs true, fractional division.
Both are executed  with the \li{/} operator.
When one or both of the operands are non-integers, \li{/} performs float division.
If both operands are integers, integer division is performed.

% TODO: for Python 3, change all operands to ints and use // for the second one
\begin{lstlisting}
>>> 15.0 / 4.0                      # Float division performs as expected.
3.75
>>> 15 / 4                          # Integer division rounds the result to 3.
3
\end{lstlisting}

\begin{warn} % TODO: This warning can be deleted for Python 3.
Using integer division unintentionally is an incredibly common programming mistake.
To avoid this, cast at least one operand as a float before dividing.
\begin{lstlisting}
>>> x, y = 7, 5
>>> x / y
1
>>> float(x) / y
1.4
>>> x / float(y)
1.4
\end{lstlisting}
\end{warn}

Python also supports complex number computations.
Use the letter $j$, not $i$, for the imaginary part.
\begin{lstlisting}
>>> x = complex(2,3)                # Create a complex number this way...
>>> y = 4 + 5j	                    # ...or this way, using j (not i).
>>> x.real                          # Access the real part of x.
2
>>> y.imag                          # Access the imaginary part of y.
5
\end{lstlisting}

\subsection*{Strings} % -------------------------------------------------------

In Python, strings are created with either single or double quotes.
To concatenate two or more strings, use the \li{+} operator between string variables or literals.
\begin{lstlisting}
>>> str1 = "Hello"
>>> str2 = 'world'
>>> my_string = str1 + " " + str2 + '!'
>>> my_string
<<'Hello world!'>>
\end{lstlisting}

We can access parts of a string using \emph{slicing}, indicated by square brackets \li{[ ]}.
Slicing syntax is \li{[start:stop:step]}.
The parameters \li{start} and \li{stop} default to the beginning and end of the string, respectively.
The parameter \li{step} defaults to 1.

\begin{lstlisting}
>>> my_string = "Hello world!"
>>> my_string[4]                # Indexing begins at 0.
<<'o'>>
>>> my_string[-1]               # Negative indices count backward from the end.
<<'!'>>

# Slice from the 0th to the 5th character (not including the 5th character).
>>> my_string[:5]
<<'Hello'>>

# Slice from the 6th character to the end.
>>> my_string[6:]
<<'world!'>>

# Slice from the 3rd to the 8th character (not including the 8th character).
>>> my_string[3:8]
<<'lo wo'>>

# Get every other character in the string.
>>> my_string[::2]
<<'Hlowrd'>>
\end{lstlisting}

\begin{problem} % Problem: Slicing with strings.
Write two new functions, called \li{first_half()} and \li{backward()}.
\begin{enumerate}
\item \li{first_half()} should accept a parameter and return the first half of it, excluding the middle character if there is an odd number of characters.
\\
(Hint: the built-in function \li{len()} returns the length of the input.)
\item The \li{backward()} function should accept a parameter and reverse the order of its characters using slicing, then return the reversed string.
\\
(Hint: The \li{step} parameter used in slicing can be negative.)
\end{enumerate}
Use IPython to quickly test your syntax for each function.
\end{problem}

\subsection*{Lists} % ---------------------------------------------------------

A Python \li{list} is created by enclosing comma-separated values with square brackets \li{[ ]}.
Entries of a list do \emph{not} have to be of the same type.
Access entries in a list with the same indexing or slicing operations used with strings.

\begin{lstlisting}
>>> my_list = ["Hello", 93.8, "world", 10]
>>> my_list
<<['Hello', 93.8, 'world!', 10]>>
>>> my_list[0]
<<'Hello'>>
>>> my_list[-2]
<<'world!'>>
>>> my_list[:2]
<<['Hello', 93.8]>>
\end{lstlisting}

Common list methods (functions) include \li{append()}, \li{insert()}, \li{remove()}, and \li{pop()}.
Consult IPython for details on each of these methods using object introspection.
\begin{lstlisting}
>>> my_list = [1, 2]                # Create a simple list of two integers.
>>> my_list.append(4)               # Append the integer 4 to the end.
>>> my_list.insert(2, 3)            # Insert 3 at location 2.
>>> my_list
[1, 2, 3, 4]
>>> my_list.remove(3)               # Remove 3 from the list.
>>> my_list.pop()                   # Remove (and return) the last entry.
4
>>> my_list
[1, 2]
\end{lstlisting}

Slicing is also very useful for replacing values in a list.

\begin{lstlisting}
>>> my_list = [10, 20, 30, 40, 50]
>>> my_list[0] = -1
>>> my_list[3:] = [8, 9]
[-1, 20, 30, 8, 9]
\end{lstlisting}

The \li{in} operator quickly checks if a given value is in a list (or a string).

\begin{lstlisting}
>>> my_list = [1, 2, 3, 4, 5]
>>> 2 in my_list
True
>>> 6 in my_list
False
\end{lstlisting}

\subsection*{Tuples} % --------------------------------------------------------

A Python \li{tuple} is an ordered collection of elements, created by enclosing comma-separated values with parentheses \li{(} and \li{)}.
Tuples are similar to lists, but they are much more rigid, have less built-in operations, and cannot be altered after creation.
Lists are therefore preferable for managing dynamic ordered collections of objects.

When multiple objects are returned in by a function, they are returned as a tuple.
For example, recall that the \li{arithmetic()} function returns two values.

\begin{lstlisting}
>>> x, y = arithmetic(5,2)                      # Get each value individually,
>>> print x, y
3 10
>>> both = arithmetic(5,2)                      # or get them both as a tuple.
>>> print(both)
(3, 10)
\end{lstlisting}

\begin{problem} % Problem: Simple list operations.
Write a function called \li{list_ops()}.
Define a list with the entries \li{"bear"}, \li{"ant"}, \li{"dog"}, and \li{"cat"}, in that order.
Then perform the following operations on the list:
\begin{enumerate}
\item Append \li{"eagle"}.
\item Replace the entry at index 2 with \li{"fox"}.
\item Remove the entry at index 1.
\item Sort the list in reverse alphabetical order.
\end{enumerate}
Return the resulting list.
\end{problem}

\subsection*{Sets} % ----------------------------------------------------------

A Python \li{set} is an unordered collection of distinct objects.
Objects can be added to or removed from a set after its creation.
Initialize a set with curly braces \li{\{ \}}, separating the values by commas, or use \li{set()} to create an empty set.

\begin{lstlisting}
# Initialize some sets. Note that repeats are not added.
>>> gym_members = {"Doe, John", "Doe, John", "Smith, Jane", "Brown, Bob"}
>>> print(gym_members)
<<set(['Smith, Jane', 'Brown, Bob', 'Doe, John'])>>

>>> gym_members.add("Lytle, Josh")      # Add an object to the set.
>>> gym_members.discard("Doe, John")    # Delete an object from the set.
>>> print(gym_members)
<<set(['Lytle, Josh', 'Smith, Jane', 'Brown, Bob'])>>
\end{lstlisting}

Like mathematical sets, Python sets have operations like union and intersection.
% difference, and symmetric difference.

\begin{lstlisting}
>>> gym_members.intersection({"Lytle, Josh", "Henriksen, Amelia" "Webb, Jared"})
<<set(['Lytle, Josh'])>>
\end{lstlisting}

\subsection*{Dictionaries} % --------------------------------------------------

% TODO: Dictionaries are a little weirder in Python 3. For example,
%   the keys() and values() methods are generators.
Like a set, a Python \li{dict} (dictionary) is an unordered data type.
A dictionary stores key-value pairs, called \emph{items}.
The values of a dictionary are indexed by its keys.
Dictionaries are initialized with curly braces, colons, and commas.
Use \li{dict()} or \li{\{\}} to create an empty dictionary.

\begin{lstlisting}
>>> my_dictionary = {"business": 4121, "math": 2061, "visual arts": 7321}
>>> print(my_dictionary["math"])
2061

# Add a value indexed by 'science' and delete the 'business' keypair.
>>> my_dictionary["science"] = 6284
>>> my_dictionary.pop("business")       # Use 'pop' or 'popitem' to remove.
4121
>>> print(my_dictionary)
<<{'visual arts': 7321, 'math': 2016, 'science': 6284}>>

# Display the keys and values.
>>> my_dictionary.keys()
<<['visual arts', 'math', 'science']>>
>>> my_dictionary.values()
<<[7321, 2016, 6284]>>
\end{lstlisting}

The keys of a dictionary must be \emph{immutable}, which means that they must be objects that cannot be modified after creation.
We will discuss mutability more thoroughly in the next lab.
% TODO: Reference the Standard Library lab.

\subsection*{Type Casting} % --------------------------------------------------

The names of each of Python's data types can be used as functions to cast a value as that type.
This is particularly useful with integers and floats.

\begin{lstlisting}
# Cast numerical values as different kinds of numerical values.
>>> x = int(3.0)
>>> y = float(3)
>>> z = complex(3)
>>> print x, y, z
3, 3.0, (3+0j)

# Cast a list as a set and vice versa.
>>> a = set([1, 2, 3, 4, 4])
>>> b = list({'a', 'a', 'b', 'b', 'c'})
>>> print(a)
<<set([1, 2, 3, 4])>>
>>> print(b)
<<['a', 'b', 'c']>>

# Cast other objects as strings.
>>> s = str(['a', str(1), 'b', float(2)])
>>> t = str(list(set([complex(float(3))]))
>>> s
<<"['a', '1', 'b', 2.0]">>
>>> t
<<'[(3+0j)]'>>

\end{lstlisting}

\section*{Control Flow Tools} % ===============================================

Control flow blocks dictate the order in which code is executed.
Python supports the usual control flow statements including \li{if} statements, \li{while} loops and \li{for} loops.

\subsection*{The If Statement} % ----------------------------------------------

An \li{if} statement executes the indented code \emph{if} (and only if) the given condition holds.
The \li{elif} statement is short for ``else if'' and can be used multiple times following an if statement, or not at all.
The \li{else} keyword may be used at most once at the end of a series of \li{if}/\li{elif} statements.

\begin{lstlisting}
>>> food = "bagel"
>>> if food == "apple":             # As with functions, the colon denotes
...     print("72 calories")        # the start of each code block.
... elif food == "banana" or food == "carrot":
...     print("105 calories")
... else:
...     print("calorie count unavailable")
...
calorie count unavailable
\end{lstlisting}

\begin{problem} % Problem: If statements and Pig Latin.
Write a function called \li{pig_latin()}.
Accept a parameter \li{word}, translate it into Pig Latin, then return the translation.
Specifically, if \li{word} starts with a vowel, add ``hay'' to the end; if \li{word} starts with a consonant, take the first character of \li{word}, move it to the end, and add ``ay''.
\\
(Hint: use the \li{in} operator to check if the first letter is a vowel.)
\end{problem}

\subsection*{The While Loop} % ------------------------------------------------

A \li{while} loop executes an indented block of code \emph{while} the given condition holds.

\begin{lstlisting}
>>> i = 0
>>> while i < 10:
...     print(i),
...     i = i+1
...
0 1 2 3 4 5 6 7 8 9
\end{lstlisting}

There are two additional useful statements to use inside of loops:
\begin{enumerate}
\item The \li{break} statement manually exits the loop, regardless of which iteration the loop is on or if the termination condition is met.
\item The \li{continue} statement skips the current iteration and returns to the top of the loop block if the termination condition is still not met.
\end{enumerate}

\begin{lstlisting}
>>> i = 0
>>> while True:
...    print(i),
...    i += 1
...    if i >= 10:
...        break                    # Exit the loop.
...
0 1 2 3 4 5 6 7 8 9

>>> i = 0
>>> while i < 10:
...     i += 1
...     if i % 3 == 0:
...         continue                # Skip multiples of 3.
...     print(i),
1 2 4 5 7 8 10
\end{lstlisting}

\subsection*{The For Loop} % --------------------------------------------------

A \li{for} loop iterates over the items in any \emph{iterable}.
Iterables include (but are not limited to) strings, lists, sets, and dictionaries.

\begin{lstlisting}
>>> colors = ["red", "green", "blue", "yellow"]
>>> for entry in colors:
...    print(entry + "!")
...
red!
green!
blue!
yellow!
\end{lstlisting}

The \li{break} and \li{continue} statements also work in for loops, but a \li{continue} in a for loop will automatically increment the index or item, whereas a \li{continue} in a while loop makes no automatic changes to any variable.

In addition, Python has some very useful built-in functions that can be used in conjunction with the \li{for} statement:
\begin{enumerate}
\item \li{range(start, stop, step)}: Produces a list of integers, following slicing syntax. If only one argument is specified, it produces a list of integers from 0 to the argument, incrementing by one.
In Python 2.7, we use \li{xrange()} more often because it is faster than \li{range()} and does not explicitly create a list in memory.
\item \li{zip()}: Joins multiple sequences so they can be iterated over simultaneously.
\item \li{enumerate()}: Yields both a count and a value from the sequence. Typically used to get both the index of an item and the actual item simultaneously.
\item \li{sorted()}: Returns a new list of sorted items that can then be used for iteration.
\item \li{reversed()}: Reverses the order of the iteration.
\end{enumerate}
% TODO: get rid of the xrange() note for Python 3.

\begin{lstlisting}
>>> vowels = "aeiou"
>>> colors = ["red", "yellow", "white", "blue", "purple"]
>>> for i in xrange(5):
...     print i, vowels[i], colors[i]
...
0 a red
1 e yellow
2 i white
3 o blue
4 u purple

>>> for letter, word in zip(vowels, colors):
...     print letter, word
...
a red
e yellow
i white
o blue
u purple

>>> for i, color in enumerate(colors):
...     print i, color
...
0 red
1 yellow
2 white
3 blue
4 purple

>>> for item in sorted(colors):
...     print item,
...
blue purple red white yellow

>>> for item in reversed(colors):
...     print item,
...
yellow white red purple blue

# Range arguments follow slicing syntax.
>>> range(10)
[0, 1, 2, 3, 4, 5, 6, 7, 8, 9]
>>> range(4, 8)
[4, 5, 6, 7]
>>> range(2, 20, 3)
[2, 5, 8, 11, 14, 17]
\end{lstlisting}

\begin{problem} % Problem: large palindromic number (Project Euler #4)
A palindromic number reads the same both ways.
The largest palindrome made from the product of two 2-digit numbers is $9009 = 91 \times 99.$\\
Write a function called \li{palindrome()} that finds and returns the largest palindromic number made from the product of two 3-digit numbers.

This problem originates from \url{https://projecteuler.net}, an excellent resource for math-related coding problems.
\end{problem}

\subsection*{List Comprehension} % --------------------------------------------

A \emph{list comprehension} uses for loop syntax between square brackets to create a list.
This is a powerful, efficient way to build lists.
The code is concise and runs quickly.

\begin{lstlisting}
>>> [float(n) for n in range(5)]
[0.0, 1.0, 2.0, 3.0, 4.0]
\end{lstlisting}

List comprehensions can be thought of as ``inverted loops'', meaning that the body of the loop comes before the looping condition.
The following loop and list comprehension produce the same list, but the list comprehension takes only about two-thirds the time to execute.

\begin{lstlisting}
>>> loop_output = []
>>> for i in range(5):
...    loop_output.append(i**2)
...
>>> list_output = [i**2 for i in range(5)]
\end{lstlisting}

Tuple, set, and dictionary comprehensions can be done in the same way as list comprehensions by using the appropriate style of brackets on the end.

\begin{lstlisting}
>>> colors = ["red", "blue", "yellow"]
>>> {c[0]:c for c in colors}
<<{'y': 'yellow', 'r': 'red', 'b': 'blue'}>>

>>> {"bright " + c for c in colors}
<<set(['bright blue', 'bright red', 'bright yellow'])>>
\end{lstlisting}

\begin{problem} % Problem: the alternating harmonic series.
The alternating harmonic series is given by the following:
\[\sum_{n=1}^\infty \frac{(-1)^{(n+1)}}{n} = 1 - \frac{1}{2} + \frac{1}{3} - \frac{1}{4} + \frac{1}{5} - \ldots = \ln(2)\]

Write a function called \li{alt_harmonic()}.
Use a list comprehension to quickly compute the first $n$ terms of this series, given the parameter \li{n}.
The sum of the first 500,000 terms of this series approximates $\ln(2)$ to five decimal places.
\\
(Hint: consider using Python's built-in \li{sum()} function.)
\end{problem}

% \section*{Conclusion}
Refer back to this and other introductory labs often as you continue getting used to Python syntax and data types.
As you continue your study of Python, we strongly recommend the following readings:
\begin{enumerate}
\item Chapters 3, 4, and 5 of the Official Python Tutorial \\
        (\url{http://docs.python.org/2.7/tutorial/introduction.html}).
\item Section 1.2 of the SciPy Lecture Notes\\
        (\url{http://scipy-lectures.github.io/}).
\item PEP8 - Python Style Guide \\
        (\url{http://www.python.org/dev/peps/pep-0008/}).
\end{enumerate}

\newpage

\section*{Additional Material} % ==============================================

\subsection*{Lambda Functions} % ----------------------------------------------

The keyword \li{lambda} is a shortcut for creating one-line functions.
For example, we can create Python functions for the polynomials $f(x) = 6x^3 + 4x^2 - x + 3$ and $g(x,y,z) = x + y^2 - z^3$ using only a single line for each definition.

\begin{lstlisting}
>>> f = lambda x: 6*x**3 + 4*x**2 - x + 3
>>> g = lambda x, y, z: x + y**2 - z**3
\end{lstlisting}

\subsection*{Generalized Function Input} % ------------------------------------

On rare occasion, it is necessary to define a function without knowing exactly what the parameters will be like or how many there will be.
This is usually done by defining the function with the parameters \li{*args} and \li{**kwargs}.
Here \li{*args} is a list of the positional arguments and \li{**kwargs} is a dictionary mapping the keywords to their argument.
This is the most general form of a function definition.

\begin{lstlisting}
>>> def report(*args, **kwargs):
...     for i, arg in enumerate(args):
...         print "Argument " + str(i) + ":", arg
...     for key in kwargs:
...         print "Keyword", key, "->", kwargs[key]
...
>>> report("TK", 421, exceptional=False, missing=True)
Argument 0: TK
Argument 1: 421
Keyword missing -> True
Keyword exceptional -> False

>>> average = lambda *args: sum(args) / float(len(args))
>>> average(4, 5, 6)
5.0
>>> average(1, 2, 3, 4, 5, 6, 7, 8, 9)
5.0
\end{lstlisting}

See \url{https://docs.python.org/2.7/tutorial/controlflow.html} for more on this topic.

\subsection*{Function Decorators} % -------------------------------------------

A \emph{function decorator} is a special function that ``wraps'' other functions.
It takes in a function as input, then usually pre-processes its inputs or post-processes its outputs.

\begin{lstlisting}
>>> def typewriter(func):
...     """Decorator for printing the type of output a function returns"""
...     def wrapper(*args, **kwargs):
...         output = func(*args, **kwargs)      # Call the decorated function.
...         print "output type:", type(output)  # Do something before finishing.
...         return output                       # Return the function output.
...     return wrapper
\end{lstlisting}

The outer function, \li{typewriter()}, returns the new function \li{wrapper()}.
Since \li{wrapper()} accepts \li{*args} and \li{**kwargs} as arguments, the input function \li{func()} could accepts any number of positional or keyword arguments.

Apply a decorator to a function by tagging the function's definition with an \textbf{@} symbol and the decorator name.

\begin{lstlisting}
>>> @typewriter
... def combine(a, b, c):
...     return a*b/float(c)
...
\end{lstlisting}

Placing the \textbf{@} tag above the definition is equivalent to adding the following line of code after the function definition:

\begin{lstlisting}
>>> combine = typewriter(combine)
\end{lstlisting}

Now when we call \li{combine()}, we are really calling \li{wrapper()}, which then calls the originally defined \li{combine()}.

\begin{lstlisting}
>>> combine(3, 4, 6)
<<output type: <type 'float'>>>
2.0
\end{lstlisting}

Function decorators can also accept arguments.
This requires another level of nesting: the outermost function must define and return a decorator that defines and returns a wrapper.

\begin{lstlisting}
>>> def repeat(times=2):
...     """Decorator for calling a function several times."""
...     def decorator(func):
...         def wrapper(*args, **kwargs):
...             for _ in xrange(times):
...                 output = func(*args, **kwargs)
...             return output
...         return wrapper
...     return decorator
...
>>> @repeat(times=3)
... def hello_world():
...     print "Hello, world!"
...
>>> hello_world()
Hello, world!
Hello, world!
Hello, world!
\end{lstlisting}

See \url{https://www.python.org/dev/peps/pep-0318/} for more details.

\begin{comment} % Overkill, since they don't know how to import yet.
Decorating a function can overwrite the original function's \emph{docstring}, the string directly below the \li{def} statement that gives a description of the function.
To remedy this problem, the \li{functools} module includes a decorator for wrappers.

\begin{lstlisting}
>>> from functools import wraps

>>> def typewriter(func):
...     """Decorator for printing the type of output a function returns"""
...     @wraps(func)                            # Add the @wraps tag.
...     def wrapper(*args, **kwargs):
...         output = func(*args, **kwargs)
...         print "output type:", type(output)
...         return output
...     return wrapper
...
>>> def repeat(times=2):
...     """Decorator for calling a function several times."""
...     def decorator(func):
...         @wraps(func)                        # Add the @wraps tag.
...         def wrapper(*args, **kwargs):
...             for _ in xrange(times):
...                 output = func(*args, **kwargs)
...             return output
...         return wrapper
...     return decorator
\end{lstlisting}
\end{comment}
