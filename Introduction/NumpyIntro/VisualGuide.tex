\lab{NumPy Visual Guide}{NumPy Visual Guide}
\label{appendix:numpy-visual-guide}
\objective{NumPy operations can be difficult to visualize, but the concepts are straightforward.
This appendix provides visual demonstrations of how NumPy arrays are used with slicing syntax, stacking, broadcasting, and axis-specific operations.
Though these visualizations are for 1- or 2-dimensional arrays, the concepts can be extended to $n$-dimensional arrays.
% See Lab \ref{lab:NumPy} for an introduction to NumPy operations and synatx.
}

\section*{Data Access} % ======================================================

The entries of a 2-D array are the rows of the matrix (as 1-D arrays).
To access a single entry, enter the row index, a comma, and the column index.
Remember that indexing begins with $0$.

\begin{align*}
A[0] = \left[\begin{array}{rrrrr}
\tikzmarkin{row1}\times & \times & \times & \times & \times\tikzmarkend{row1}\\
\times & \times & \times & \times & \times\\
\times & \times & \times & \times & \times\\
\times & \times & \times & \times & \times\\
\end{array}\right]
&&
A[2,1] = \left[\begin{array}{rrrrr}
\times & \times & \times & \times & \times\\
\times & \times & \times & \times & \times\\
\times & \tikzmarkin{entry} \times \tikzmarkend{entry} & \times & \times & \times\\
\times & \times & \times & \times & \times
\end{array}\right]
\end{align*}

\section*{Slicing} % ==========================================================

A lone colon extracts an entire row or column from a 2-D array.
The syntax \li{[a:b]} can be read as ``the $a^{th}$ entry up to (but not including) the $b^{th}$ entry.''
Similarly, \li{[a:]} means ``the $a^{th}$ entry to the end'' and \li{[:b]} means ``everything up to (but not including) the $b^{th}$ entry.''

\begin{align*}
\text{\li{A[1]}} = \text{\li{A[1,:]}} = \left[\begin{array}{rrrrr}
\times & \times & \times & \times & \times\\
\tikzmarkin{row2}\times & \times & \times & \times & \times\tikzmarkend{row2}\\
\times & \times & \times & \times & \times\\
\times & \times & \times & \times & \times\\
\end{array}\right]
&&
\text{\li{A[:,2]}} = \left[\begin{array}{rrrrr}
\times & \times & \tikzmarkin{col}\times & \times & \times\\
\times & \times & \times & \times & \times\\
\times & \times & \times & \times & \times\\
\times & \times & \times\tikzmarkend{col} & \times & \times
\end{array}\right]
\end{align*}

\begin{align*}
\text{\li{A[1:,:2]}} = \left[\begin{array}{rrrrr}
\times & \times & \times & \times & \times\\
\tikzmarkin{block}\times & \times & \times & \times & \times\\
\times & \times & \times & \times & \times\\
\times & \times\tikzmarkend{block} & \times & \times & \times
\end{array}\right]
&&
\text{\li{A[1:-1,1:-1]}} = \left[\begin{array}{rrrrr}
\times & \times & \times & \times & \times\\
\times & \tikzmarkin{interior} \times & \times & \times & \times\\
\times & \times & \times & \times \tikzmarkend{interior} & \times\\
\times & \times & \times & \times & \times\end{array}\right]
\end{align*}

\section*{Stacking} % =========================================================

\li{np.hstack()} stacks sequence of arrays horizontally and \li{np.vstack()} stacks a sequence of arrays vertically.

\begin{align*}
A = \left[\begin{array}{ccc}
\textcolor{blue}{\times}&\textcolor{blue}{\times}&\textcolor{blue}{\times}\\
\textcolor{blue}{\times}&\textcolor{blue}{\times}&\textcolor{blue}{\times}\\
\textcolor{blue}{\times}&\textcolor{blue}{\times}&\textcolor{blue}{\times}
\end{array}\right]
&&
B = \left[\begin{array}{ccc}
\textcolor{red}{*} & \textcolor{red}{*} & \textcolor{red}{*} \\
\textcolor{red}{*} & \textcolor{red}{*} & \textcolor{red}{*} \\
\textcolor{red}{*} & \textcolor{red}{*} & \textcolor{red}{*}
\end{array}\right]
\end{align*}

\begin{align*}
\text{\li{np.hstack((A,B,A))}} =
\left[\begin{array}{ccccccccc}
\textcolor{blue}{\times}&\textcolor{blue}{\times}&\textcolor{blue}{\times}&
\textcolor{red}{*} & \textcolor{red}{*} & \textcolor{red}{*}&
\textcolor{blue}{\times}&\textcolor{blue}{\times}&\textcolor{blue}{\times}\\
\textcolor{blue}{\times}&\textcolor{blue}{\times}&\textcolor{blue}{\times}&
\textcolor{red}{*} & \textcolor{red}{*} & \textcolor{red}{*}&
\textcolor{blue}{\times}&\textcolor{blue}{\times}&\textcolor{blue}{\times}\\
\textcolor{blue}{\times}&\textcolor{blue}{\times}&\textcolor{blue}{\times}&
\textcolor{red}{*} & \textcolor{red}{*} & \textcolor{red}{*}&
\textcolor{blue}{\times}&\textcolor{blue}{\times}&\textcolor{blue}{\times}
\end{array}\right]
\end{align*}

\begin{align*}
\text{\li{np.vstack((A,B,A))}} =
\left[\begin{array}{ccc}
\textcolor{blue}{\times}&\textcolor{blue}{\times}&\textcolor{blue}{\times}\\
\textcolor{blue}{\times}&\textcolor{blue}{\times}&\textcolor{blue}{\times}\\
\textcolor{blue}{\times}&\textcolor{blue}{\times}&\textcolor{blue}{\times}\\
\textcolor{red}{*} & \textcolor{red}{*} & \textcolor{red}{*} \\
\textcolor{red}{*} & \textcolor{red}{*} & \textcolor{red}{*} \\
\textcolor{red}{*} & \textcolor{red}{*} & \textcolor{red}{*} \\
\textcolor{blue}{\times}&\textcolor{blue}{\times}&\textcolor{blue}{\times}\\
\textcolor{blue}{\times}&\textcolor{blue}{\times}&\textcolor{blue}{\times}\\
\textcolor{blue}{\times}&\textcolor{blue}{\times}&\textcolor{blue}{\times}
\end{array}\right]
\end{align*}
Because 1-D arrays are flat, \li{np.hstack()} concatenates 1-D arrays and \li{np.vstack()} stacks them vertically.
To make several 1-D arrays into the columns of a 2-D array, use \li{np.column_stack()}.

\begin{align*}
x = \left[\begin{array}{cccc}
\textcolor{blue}{\times}&\textcolor{blue}{\times}&\textcolor{blue}{\times}&\textcolor{blue}{\times}
\end{array}\right]
&&
y = \left[\begin{array}{cccc}
\textcolor{red}{*}&\textcolor{red}{*}&\textcolor{red}{*}&\textcolor{red}{*}\end{array}\right]
\end{align*}

\begin{align*}
\text{\li{np.hstack((x,y,x))}} =
\left[\begin{array}{cccccccccccc}
\textcolor{blue}{\times}&\textcolor{blue}{\times}&\textcolor{blue}{\times}&\textcolor{blue}{\times}&
\textcolor{red}{*}&\textcolor{red}{*}&\textcolor{red}{*}&\textcolor{red}{*}&
\textcolor{blue}{\times}&\textcolor{blue}{\times}&\textcolor{blue}{\times}&\textcolor{blue}{\times}
\end{array}\right]
\end{align*}

\begin{align*}
\text{\li{np.vstack((x,y,x))}} =
\left[\begin{array}{cccc}
\textcolor{blue}{\times}&\textcolor{blue}{\times}&\textcolor{blue}{\times}&\textcolor{blue}{\times}\\
\textcolor{red}{*}&\textcolor{red}{*}&\textcolor{red}{*}&\textcolor{red}{*}\\
\textcolor{blue}{\times}&\textcolor{blue}{\times}&\textcolor{blue}{\times}&\textcolor{blue}{\times}
\end{array}\right]
&&
\text{\li{np.column_stack((x,y,x))}} =
\left[\begin{array}{ccc}
\textcolor{blue}{\times}&\textcolor{red}{*}&\textcolor{blue}{\times}\\
\textcolor{blue}{\times}&\textcolor{red}{*}&\textcolor{blue}{\times}\\
\textcolor{blue}{\times}&\textcolor{red}{*}&\textcolor{blue}{\times}\\
\textcolor{blue}{\times}&\textcolor{red}{*}&\textcolor{blue}{\times}
\end{array}\right]
\end{align*}

\section*{Broadcasting} % =====================================================

NumPy automatically aligns arrays for component-wise operations whenever possible.
% The default behavior adds the first element to the first element of each row, the second element to the second element of each row, and so on.
See \url{http://docs.scipy.org/doc/numpy/user/basics.broadcasting.html} for more in-depth examples and broadcasting rules.

\begin{align*}
A = \left[\begin{array}{ccc}
1 & 2 & 3\\
1 & 2 & 3\\
1 & 2 & 3\\
\end{array}\right]
&&
x = \left[\begin{array}{ccc}
10 & 20 & 30
\end{array}\right]
\end{align*}

\begin{align*}
\text{\li{A + x}}
&= \begin{blockarray}{ccc}
\begin{block}{[ccc]}
1 & 2 & 3\\
1 & 2 & 3\\
1 & 2 & 3\\
\end{block}
  & + &  \\
\begin{block}{[ccc]}
10 & 20 & 30\\
\end{block}
\end{blockarray}
&= \left[\begin{array}{ccc}
11 & 22 & 33\\
11 & 22 & 33\\
11 & 22 & 33
\end{array}\right]
\\ \\
\text{\li{A + np.vstack(x)}}
&= \left[\begin{array}{ccc}
1 & 2 & 3 \\
1 & 2 & 3 \\
1 & 2 & 3 \\
\end{array}\right]
+ \left[\begin{array}{c}
10 \\ 20 \\ 30\\
\end{array}\right]
&= \left[\begin{array}{ccc}
11 & 12 & 13\\
21 & 22 & 23\\
31 & 32 & 33\\
\end{array}\right]
\end{align*}

\section*{Operations along an Axis} % =========================================

Most array methods have an \li{axis} argument that allows an operation to be done along a given axis.
To compute the sum of each column, use \li{axis=0}; to compute the sum of each row, use \li{axis=1}.

\begin{align*}
A = \left[\begin{array}{cccc}
1 & 2 & 3 & 4\\
1 & 2 & 3 & 4\\
1 & 2 & 3 & 4\\
1 & 2 & 3 & 4
\end{array}\right]
\end{align*}

\begin{align*}
\text{\li{A.<<sum>>(axis=0)}} &= %np.array([sum(A[:,i] for i in xrange(A.shape[1]))])}} =
\left[\begin{array}{cccc}
\tikzmarkin{col1}1 & \tikzmarkin{col2}2 & \tikzmarkin{col3}3 & \tikzmarkin{col4}4\\
1 & 2 & 3 & 4\\
1 & 2 & 3 & 4\\
1\tikzmarkend{col1} & 2\tikzmarkend{col2} & 3\tikzmarkend{col3} & 4\tikzmarkend{col4}
\end{array}\right]
= \left[\begin{array}{cccc} 4 & 8 & 12 & 16 \end{array}\right]
\\ \\
\text{\li{A.<<sum>>(axis=1)}} &= %np.array([sum(A[i,:] for i in xrange(A.shape[0]))])}} =
\left[\begin{array}{cccc}
\tikzmarkin{rowA}1 & 2 & 3 & 4\tikzmarkend{rowA}\\
\tikzmarkin{rowB}1 & 2 & 3 & 4\tikzmarkend{rowB}\\
\tikzmarkin{rowC}1 & 2 & 3 & 4\tikzmarkend{rowC}\\
\tikzmarkin{rowD}1 & 2 & 3 & 4\tikzmarkend{rowD}\\
\end{array}\right]
= \left[\begin{array}{cccc} 10 & 10 & 10 & 10 \end{array}\right]
\end{align*}

